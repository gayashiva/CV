%%%%%%%%%%%%%%%%%%%%%%%%%%%%%%%%%%%%%%%%%
% "ModernCV" CV and Cover Letter
% LaTeX Template
% Version 1.11 (19/6/14)
%
% This template has been downloaded from:
% http://www.LaTeXTemplates.com
%
% Original author:
% Xavier Danaux (xdanaux@gmail.com)
%
% License:
% CC BY-NC-SA 3.0 (http://creativecommons.org/licenses/by-nc-sa/3.0/)
%
% Important note:
% This template requires the moderncv.cls and .sty files to be in the same 
% directory as this .tex file. These files provide the resume style and themes 
% used for structuring the document.
%
%%%%%%%%%%%%%%%%%%%%%%%%%%%%%%%%%%%%%%%%%

%----------------------------------------------------------------------------------------
%	PACKAGES AND OTHER DOCUMENT CONFIGURATIONS
%----------------------------------------------------------------------------------------

\documentclass[11pt,a4paper,sans]{moderncv} % Font sizes: 10, 11, or 12; paper sizes: a4paper, letterpaper, a5paper, legalpaper, executivepaper or landscape; font families: sans or roman

\moderncvstyle{casual} % CV theme - options include: 'casual' (default), 'classic', 'oldstyle' and 'banking'

\moderncvcolor{blue} % CV color - options include: 'blue' (default), 'orange', 'green', 'red', 'purple', 'grey' and 'black'

\usepackage[scale=0.75]{geometry} % Reduce document margins

%\setlength{\hintscolumnwidth}{3cm} % Uncomment to change the width of the dates column

%\setlength{\makecvtitlenamewidth}{10cm} % For the 'classic' style, uncomment to adjust the width of the space allocated to your name

%\usepackage{hyperref}

%\urlstyle{same}


%----------------------------------------------------------------------------------------
%	NAME AND CONTACT INFORMATION SECTION
%----------------------------------------------------------------------------------------

\familyname{\color{black}Suryanarayanan} % Your first name

\firstname{\color{black}Balasubramanian} % Your last name


% All information in this block is optional, comment out any lines you don't need
\title{Curriculum Vitae}

%\address{30/7 Kartik Apartments}{Andavar Koil Street, Vadapalani,Chennai 600026}

\mobile{(+91) 9840803518}

\email{gayashiva91@gmail.com}

%\photo[70pt][0.4pt]{pictures/15998} % The first bracket is the picture height, the second is the thickness of the frame around the picture (0pt for no frame)

%----------------------------------------------------------------------------------------

\begin{document}

\makecvtitle % Print the CV title

%----------------------------------------------------------------------------------------
%	WORK EXPERIENCE SECTION
%----------------------------------------------------------------------------------------

\section{Work Experience}

\cventry{September 2018-Present}{Artificial Ice Reservoirs}{PhD}{}{}{Developing a weather based physical model
to predict meltwater discharge and survival duration of Artificial Ice Reservoirs (AIRs, also called icestupas).
This PhD project is funded via the Swiss Government Excellence Scholarship } \begin{flushright}
  \href{https://www.frontiersin.org/articles/10.3389/feart.2021.771342/full}{[{\color{blue}View Paper}]}
\end{flushright}

\cventry{August 2015--March 2019}{Project Manager}{\href{http://icestupa.org/}{\color{blue}\textsc{Ice Stupa
  Artificial Glaciers}}}{}{}{Project goal is to store wasting winter water in the form of ice mountains that
  melt and feed the farms when water is most needed by the farmers. I documented the projects activities through
  films for dissemination in the media. As a researcher, my contribution was to collect data of various
  environmental variables and analyse them. Also I designed pipeline systems to achieve the desired discharge
for various ice stupa implementations. I also implemented the project in Switzerland and Peru as part of our
outreach acivities.} \begin{flushright} \href{https://www.facebook.com/icestupa/}{[{\color{blue}View Project
News}]} \end{flushright}

\cventry{August --September 2016}{GLOF risk
reduction}{\href{http://www.ssdma.nic.in/}{\color{blue}\textsc{Sikkim State Disaster Management
Agency}}}{}{}{Part of the team of 4 selected to lead the expedition aimed to mitigate risk of a potential
Glacial Lake Outburst Flood(GLOF) by draining the lake using siphon technique.} \begin{flushright}
\href{http://icestupa.org/news/slug-rb4s71}{[{\color{blue}View Blog}]} \end{flushright}

% \cventry{April 2017--November 2017}{Lake Ice Model}{{\textsc{St. Moritz, Switzerland}}}{}{}{Recreated an IDL
% model in python to predict the St. Moritz lake ice thickness for a period of one week. This is to ensure the
% strength of the lake ice to conduct events on it. } \begin{flushright}
% \href{https://www.dropbox.com/s/w5skbziw4aex0if/lakeice_sample.pdf?dl=0}{[{\color{blue}Sample Lake Ice
% Simulation}]} \end{flushright}

% \cventry{April 2017--November 2017}{Baby Glacier Experiment}{{\textsc{Morteratsch Glacier,
% Switzerland}}}{}{}{Summer storage of snow has seen an increasing interest in the last years. This paper
% attempts to validate the findings using mass balance measurements of such an artificially sustained snow pile
% obtained by manual measurements during summer 2017. } \begin{flushright}
% \href{https://www.dropbox.com/s/nqcbjs2du35tcjk/Baby_glacier_report.pdf?dl=0}{[{\color{blue}View Report}]}
% \end{flushright}

% \cventry{August 2017}{Icestupa Project Implementation}{{\textsc{Quelccaya Ice Cap, Peru}}}{}{}{Constructed ice
% structures and tested Ice Stupa project Equipments }

% \cventry{October 2017--November 2017}{Icestupa Project Implementation}{{\textsc{Morteratsch
% Glacier}}}{}{}{Guided construction of Icestupa in collaboration with Prof. Johannes Oerlemans and Dr. Felix
% Keller }


%\cventry{November 2016--February 2017}{Campaign
%Manager}{\href{http://hial.co.in/}{\color{blue}\textsc{Himalayan Institute of Alternatives}}}{}{}{Part of the
%team establishing the first mountain university in the Ladakh region of the Himalayas. Created a crowdfunding
%campaign video and oversaw its publicity activities.} \begin{flushright}
%\href{https://milaap.org/fundraisers/hial}{[{\color{blue}View Campaign}]} \end{flushright}
%
%\cventry{December 2016}{Artificial Glacier Growth Analysis}{{\textsc{Python}}}{}{}{Analysis of the ice stupa's
%growth rate as a function of diverse environmental variables in collaboration with Dr. Nicolas Perony. This is
%a work in progress.} \begin{flushright}
%\href{https://github.com/slowmotionprojects/icestupa/blob/master/icestupa_growth_analysis/notebooks/2.0-np-analysis.ipynb}{[{\color{blue}View
%Github Project}]} \end{flushright}
%
%
%\cventry{April 2016--October 2016}{Icestupa Project Implementation}{{\textsc{Morteratsch}}}{}{}{Guided
%construction of Icestupa in collaboration with Dr. Johannes Oerlemans and Dr. Felix Keller }
%

% \cventry{April 2018}{Artificial Glacier Growth Analysis}{{\textsc{Python}}}{}{}{In order to optimise the
% growth of the ice formation throughout the winter, and understand its melt in the spring and summer, I have
% created a data analysis and modelling pipeline that correlates meteorological parameters recorded to the
% observed height growth rate in timelapse imagery.  } \begin{flushright}
% \href{https://github.com/Gayashiva/Val_Roseg}{[{\color{blue}View Github Project}]} \end{flushright}

%\cventry{April 2018-July 2018}{The Palar Center for Learning and its education for
%sustainability}{{}}{}{}{Documented sustainable practices for assessing impact and creating guidelines for other
%schools } \begin{flushright}
%\href{https://drive.google.com/file/d/1GYMjsgyhWVZzW7j5PDCsCwFdGfN315SD/view?usp=sharing}{[{\color{blue}View
%Report}]} \end{flushright}
%
%\cventry{August 2016}{Siphon Technique Study}{{\LaTeX}}{}{}{A brief report on the siphon technique aimed at
%understanding its limitations.} \begin{flushright}
%\href{https://drive.google.com/open?id=0B2M2W1P78he-OFFRSFRsS1FOaHM}{[{\color{blue}View Document}]}
%\end{flushright}
%
%\cventry{December 2015}{Simulation Model}{\href{http://icestupa.org/}{\textsc{JAVA}}}{}{}{Created a temperature
%based model simulating the icestupa's growth.} \begin{flushright}
%\href{https://drive.google.com/open?id=0B2M2W1P78he-dXBrcDlKelhFV0U}{[{\color{blue}View Documentation}]}
%\end{flushright}
%
%\cventry{June 2015}{Power GEMS School}{}{}{}{Volunteering to install Solar Panels in the school to solve its
%energy crisis} \begin{flushright} \href{https://vimeo.com/136603051}{[{\color{blue}View Campaign Video}]}
%\end{flushright} \cventry{Dec 2014--May
%2015}{President}{}{\href{www.zaariya.org}{\color{blue}\textsc{Zaariya}}}{}{ This club was founded due to my
%success in rehabilitating a child labourer and sending him back to school. }
%
%\begin{flushright} \href{https://www.facebook.com/Zariya.NISER?ref=aymt_homepage_panel}{[{\color{blue}View
%Zaariya's facebook page}]} \end{flushright}
%
%\cventry{Aug 2014--March 2015}{Volunteer Skype
%Teacher}{\href{http://www.evidyaloka.org/}{\color{blue}\textsc{evidyaloka}}}{}{}{I taught 7th grade Maths in
%hindi at a government school in Jharkhand.The focus is to improve the learning outcome of the children in these
%remote locations through the digital class rooms established.}
%
%\begin{flushright}
%\href{https://medium.com/@suryanarayananbalasubramanian/right-now-all-over-rural-india-this-is-happening-843d44ad38f1}{[{\color{blue}View
%Blog}]} \end{flushright}


%---------------------------------------------------------------------------------------- EDUCATION SECTION
%----------------------------------------------------------------------------------------

\section{Education}

\cventry{September 2017}{Karthaus School}{Sudtirol, Italy}{}{Summerschool on Ice Sheets and Glaciers in the
Climate System} {\href{https://www.projects.science.uu.nl/iceclimate/karthaus/}{[\color{blue}{View Course}]}}

\cventry{April 2017--July 2017}{Machine Learning}{{\textsc{Coursera}}}{}{}{This course provides a broad
introduction to machine learning, datamining, and statistical pattern recognition. Passed it with a grade of
95.3 }

\cventry{July 2015--August 2016}{India Fellow}{Ladakh}{}{Placed  as an apprentice to SECMOL founder Sonam
Wangchuk.} {\href{https://indiafellow.wordpress.com/author/gayashiva/}{[\color{blue}{View Blog}]}}

\cventry{Aug 2010--May 2015}{Integrated M.Sc.(Maths)}{National Institute of Science Education and
Research}{Bhubaneswar}{6.27/10}{\href{https://drive.google.com/open?id=0B2M2W1P78he-ckdxV3lEWmVsTDg}{[{\color{blue}View
Thesis}]}}  

\cventry{2007--2009}{Central board of Secondary Education(Class 12)}{Kendriya Vidyalaya Ashok
Nagar}{Chennai}{\textit{81.8 \%}}{}

\cventry{2005--2007}{Central board of Secondary Education(Class 10)}{Kendriya Vidyalaya Ashok
Nagar}{Chennai}{\textit{90.2 \% }}{}

\cventry{2004--2005}{Central board of Secondary Education}{Jawahar Vidyalaya}{Chennai}{}{}

\cventry{1995--2004}{Indian Certificate of Secondary Education}{Smt.Sulochanadevi Singhania School}{Thane}{}{}

%------------------------------------------------------------------

\section{Computer}

\cvitem{Programming Languages}{\textsc{Python}, \textsc{C++}, \textsc{CSS}, \textsc{HTML}, \textsc{JAVA} }

\cvitem{Packages}{ Jupyter, AnyLogic,\LaTeX, Google Earth, FinalCutPro, Wordpress}

%---------------------------------------------------------------------------------------- LANGUAGES SECTION
%----------------------------------------------------------------------------------------

\section{Languages}

\cvitemwithcomment{}{Tamil, English, Hindi}{Conversationally fluent}

%\cvitem{2014}{\textbf{General GRE} score of \textbf{321/340} \newline{}(Quantitative:161/170,80
%percentile;Verbal:160/170,84 percentile)}

%\cvitem{2014}{\textbf{TOEFL} iBT score of
%\textbf{106/120}\newline{}(Reading:28;Listening:28;Speaking:22;Writing:28)}

%---------------------------------------------------------------------------------------- AWARDS SECTION
%----------------------------------------------------------------------------------------



\section{Recognitions and Awards}

\cventry{April 2019}{European Geosciences Union}{{\textsc{Vienna}}}{}{}{Presented poster about project
activities and research proposal } \begin{flushright}
\href{https://www.dropbox.com/s/ytw4mot5261cjin/AIR.pdf?dl=0}{[{\color{blue}View Poster}]} \end{flushright}

\cvitem{2018}{Climate Action Challenge
Winner{\href{https://challenge.whatdesigncando.com/projects/artificial-glaciers-receding-mountain-woes/}{{\color{blue}--Project
page}}}}

\cventry{August 2017}{World Water Week}{{\textsc{Stockholm}}}{}{}{Managed a showcase event and present poster
about the MortAlive and Ice Stupa projects } \begin{flushright}
\href{https://www.dropbox.com/s/ahoqs2dkysjw6nx/WWWPoster_200X80.pdf?dl=0}{[{\color{blue}View Poster}]}
\end{flushright}

\cvitem{2016}{Rolex Awards for
Enterprise{\href{http://www.rolexawards.com/40/laureate/sonam-wangchuk}{{\color{blue}--Project page}}}}

% \cvitem{2015}{Featured in The
% Telegraph--{\href{https://www.facebook.com/Zariya.NISER/photos/a.750252265071600.1073741828.750240208406139/798596826903810/?type=1&theater}{{\color{blue}17th
% of April 2015 Issue}}}}

\cvitem{2014}{{\href{http://csirhrdg.res.in/jrfsrfra2.htm}{{\color{blue}Council of Scientific and Industrial
Research--University Grants Commission(CSIR--UGC) Junior Research Fellow}(Mathematics,AIR 67)}}}

%\cvitem{2013}{Second Prize--{Squint National Short film Competition\hspace{27 mm} }}

\cvitem{2010}{\href{http://www.inspire-dst.gov.in/}{{\color{blue}INSPIRE}(Innovation in Science Pursuit for
Inspired Research) Fellow}}

%\cvitem{}{Ranked 300 in \href{http://www.nestexam.in/}{{\color{blue}National Entrance Screening Test}}}

%\cvitem{}{State Rank 28 (Tamilnadu) in \href{http://www.nestexam.in/}{{\color{blue}All India Engineering
%Entrance Examination}}}

\section{Referees}

\cventry{}{Martin Hoelzle}{Professor}{\href{https://www.researchgate.net/profile/Martin_Hoelzle}{\color{blue}
University of Fribourg}}{}{Email:\href{mailto:martin.hoelzle@unifr.ch}{martin.hoelzle@unifr.ch}}

\cventry{}{Johannes Oerlemans}{Professor}{\href{https://www.uu.nl/staff/JOerlemans}{\color{blue}\textsc{Utrecht
University}}{ }}{}{Email:\href{mailto:j.oerlemans@uu.nl}{j.oerlemans@uu.nl}}

\cventry{}{Sonam Wangchuk}{Founder}{\href{http://www.secmol.org/index.php}{\color{blue}\textsc{SECMOL}}{ Ice
Stupa Project}}{}{Email:\href{mailto:sonamsolar@gmail.com}{sonamsolar@gmail.com}}

%\section{Other Projects and Workshops}

%\cventry{August 2015-September 2015}{Architect}{Third Pole}{}{}{Designed a solar passive winter resort}
%\begin{flushright}
%\href{https://drive.google.com/file/d/0B2M2W1P78he-RVkwek5OdnYwbzA/view?usp=sharing}{[{\color{blue}View
%Design}]} \end{flushright}
%
%\cventry{August 2015}{Earth Construction
%Workshop}{\href{http://www.secmol.org/index.php}{\color{blue}\textsc{SECMOL}}}{}{}{Designed and constructed a
%passive solar dormitory in this workshop.} \cventry{June 2015}{Power GEMS School}{}{}{}{Volunteering to install
%Solar Panels in the school to solve its energy crisis} \begin{flushright}
%\href{https://vimeo.com/136603051}{[{\color{blue}View Campaign Video}]} \end{flushright} \cventry{Dec 2014--May
%2015}{President}{}{\href{www.zaariya.org}{\color{blue}\textsc{Zaariya}}}{}{ This club was founded due to my
%success in rehabilitating a child labourer and sending him back to school. }
%
%\cventry{Aug 2014--March 2015}{Volunteer Skype
%Teacher}{\href{http://www.evidyaloka.org/}{\color{blue}\textsc{evidyaloka}}}{}{}{I taught 7th grade Maths in
%hindi at a government school in Jharkhand.The focus is to improve the learning outcome of the children in these
%remote locations through the digital class rooms established.}
%
%\begin{flushright}
%\href{https://medium.com/@suryanarayananbalasubramanian/right-now-all-over-rural-india-this-is-happening-843d44ad38f1}{[{\color{blue}View
%Blog}]} \end{flushright}
%


%\section{Skills}

%I facilitated the growth of the club by: \begin{itemize} \item coordinating all the volunteers and assigning
%the kids they took responsibility for.  \item organising field trips for all the volunteers and their kids.
%\item organising and updating volunteer and children information.  \item addressing day to day challenges faced
%by the volunteers and hearing out suggestions and concerns of the community about their children’s education.
%\item publicising the initiative by creating a site and updating our activities in our Facebook page.  \item
%creating awareness about the prevalent situation by reaching out to schools and government officials.  \item
%organising fundraisers to realise the goals of the initiative.  \end{itemize}

%\begin{flushright} \href{https://www.facebook.com/Zariya.NISER?ref=aymt_homepage_panel}{[{\color{blue}View
%Zaariya's facebook page}]} \end{flushright}



%\subsection{Organizational}

%\cvlistitem{Player coach of NISER 2013 football team}

%\cvlistitem{Organized event Crime Scene Investigation in 2012 and 2013 college cultural festival}

%\subsection{Misc.} \cvlistitem{Played Guitar and Mridangam}

%---------------------------------------------------------------------------------------- INTERESTS SECTION
%----------------------------------------------------------------------------------------

%\section{Interests}

%\renewcommand{\listitemsymbol}{-~} % Changes the symbol used for lists

%\cvlistdoubleitem{Football}{Volleyball} \cvlistdoubleitem{Tennis}{Animation}


%\cventry{Dec 26 --Jan 3 2015}{Workshop on Jeevan Vidya}{}{\textsc{Auroville}}{}{} { A Jeevan Vidya workshop is
%an intensive 40-hour learning experience that seeks to bring one's attention to neglected and subtle facets of
%life; issues related to interpersonal relations, education, society, environment, aspirations, success are
%discussed and participants are provided critical tools to help them explore the rich web of connections between
%seemingly disparate aspects of life. It is a process of guided introspection, of 'doing philosophy' rather than
%studying it. There is no sermonizing; the facilitator presents sets of proposals, and helps participants bring
%their attention to bear on the inner workings of their thoughts, fears and aspirations. Gradually one begins to
%interrogate hidden assumptions and get a sharper, clearer view of the whole intricate fabric of life; one
%begins to see new possibilities for positive human action. The idea is to trigger an empowering, self-critical
%inner dialogue that begins with the workshop, but doesn't end with it...}

%\cventry{Nov 8--10 2014}{Alternative education for social and political
%change}{\href{http://www.sambhaavnaa.org/}{\color{blue}\textsc{Sambhaavnaa Institute}}}{}{}{This workshop
%discussed aims of education, questions of pedagogy, in the social and geographical context of children, and
%inculcating new values, Consequences of privatization of education on various sections of society, Tyranny of
%the school: Structure, curriculum, Pedagogy, hidden agenda of schooling, promote hierarchy; religious and
%cultural homogenisation; mainstream ideas, Examples and possibilities of interventions; Challenges and
%experiences in Alternative Education}

%\cventry{Aug --Nov 2014}{Weierstrass functions}{Brundaban Sahu}{National Institute of Science Education and
%Research}{}{}\begin{flushright}\href{https://drive.google.com/file/d/0B2M2W1P78he-SG5YMUw0MnkyUlU/view?usp=sharing}{[{\color{blue}View
%Report}]}\end{flushright}

%\cventry{Oct 2--5 2014}{Mindfulness in Education
%Retreat}{\href{http://www.ahimsatrust.org/}{\color{blue}\textsc{Ahimsa Trust}}}{The Doon School}{}{This
%workshop focused on bringing mindfulness practices to teachers so that they can use it in their own lives and,
%based on their experience, share it with their students. The workshop included talks, questions and answers,
%group discussions, guided meditations, exercises in stress reduction, relaxation, mindful consumption, and the
%practice of techniques to restore and maintain good communication. There were a number of exercises shared
%which could be taught back in the classrooms and school environments.}

%\cventry{July 2014}{Volunteer Teacher}{}{\href{http://gems-school.org/}{\color{blue}{Garhwal English Medium
%School}}}{}{I set up a science lab in the school and was a stand in teacher during my
%stay.}\begin{flushright}\href{https://medium.com/@suryanarayananbalasubramanian/g-e-m-s-14d6047211f6}{[{\color{blue}View
%Blog}]}\end{flushright}

%\cventry{Jan--Apr 2014}{Elliptic Curves}{Brundaban Sahu}{National Institute of Science Education and
%Research}{}{}\begin{flushright}\href{https://drive.google.com/file/d/0B2M2W1P78he-MHUzRHIxTTF1d0k/view?usp=sharing}{[{\color{blue}View
%Report}]}\end{flushright}

%\cventry{Aug--Nov 2013}{Polynomial Rings and Affine Geometry}{Santosh Pattanayak}{Indian Institute of
%Technology}{Kanpur}{}\begin{flushright}\href{https://drive.google.com/file/d/0B2M2W1P78he-c1BBZEV2cm1UT0U/view?usp=sharing}{[{\color{blue}View
%Report}]}\end{flushright}

%------------------------------------------------

%\cventry{Summer 2013}{Commutative Algebra}{Balwant Singh}{Center for Basic Excellence in
%Sciences}{}{}\begin{flushright}\href{https://drive.google.com/file/d/0B2M2W1P78he-NnNZaWlVSlgxOE0/view?usp=sharing}{[{\color{blue}View
%Report}]}\end{flushright}

%\cventry{Summer 2012}{Thematic summer programme on finite fields}{}{Institute of Mathematical Sciences}{}{At
%the conclusion of the workshop,I wrote a short article on Quadratic residues collaborating with a
%friend}\begin{flushright}\href{https://drive.google.com/file/d/0B2M2W1P78he-MlViRERQSm02cW8/view?usp=sharing}{[{\color{blue}View
%Report}]}\end{flushright}

%\cventry{Summer 2011}{Hopf Fibrations}{Varadharajan Muruganandam }{National Institute of Science Education and
%Research}{}{}

%\cventry{Jan--Apr 2011}{Sudoku Solver(C++)}{Shashikant C. Phatak }{Center for Basic Excellence in Sciences}{}{}

%------------------------------------------------



\end{document}
